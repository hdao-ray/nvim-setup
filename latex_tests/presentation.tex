\documentclass{beamer}
\usepackage{graphicx, wrapfig, pifont, array, amsmath, amsthm, mathtools, amssymb, faktor, xfrac, setspace, physics}
\usepackage{tikz-cd}

\renewcommand\qedsymbol{\(\blacksquare\)}

\DeclareMathOperator{\adj}{adj}
\DeclareMathOperator{\ord}{ord}
\DeclareMathOperator{\lcm}{lcm}
\DeclareMathOperator{\ima}{Im}

\newcommand{\eps}{\varepsilon}
\newcommand{\phv}{\varphi}
\newcommand{\tmod}[1]{~(\text{mod}~#1)}
\newcommand{\ang}[1]{\langle #1 \rangle}

\theoremstyle{definition}
\newtheorem*{df}{Definition}

%\newcolumntype{C}{>{\(}c<{\)}}

\usetheme{metropolis}
\title{Algebraic Geometry With \(YZ = X^2\)}
\author{Ray Hu}
\date{}
\begin{document}
\maketitle
\begin{frame}
    \frametitle{Projective Space}
How do we describe the solutions to \(YZ = X^2\) in \(\mathbb{C}\)?
      \begin{itemize}
          \item[\textrightarrow] if \((X, Y, Z) = (a, b, c)\) is a solution, then \(\lambda (a, b, c)\) is also a solution for all \(\lambda \in \mathbb{C}\)
          \item[\textrightarrow] it seems each solution corresponds to a solution set...
        \item[\textrightarrow] we will see these solution sets can be described as equivalence classes

        \end{itemize}

    \end{frame}
    \begin{frame}

        \frametitle{Projective Space 2}
        Define the equivalence relation \(\sim\) on nonzero \(3\)-tuples in \(\mathbb{C}\) as \((x_0, x_1, x_{2}) \sim (y_0, y_1, y_{2})\) if and only if there exists \(\lambda \in \mathbb{C}^\ast\) such that \(x_i = \lambda y_i\) for all \(0\leq i\leq 2\).
        \begin{itemize}
            \item[\textrightarrow] Then \(\mathbb{P}^2(\mathbb{C})\), or \(\mathbb{P}^2\), is the set of these equivalence classes. Denote the equivalence classes of \((x_0, x_1, x_2)\) by \([x_0: x_1: x_2]\)
 \end{itemize}

 All solutions to \(YZ = X^2\) that are not \((0, 0, 0)\) can be described by points in \(\mathbb{P}^2\).
\end{frame}

\begin{frame}
    \frametitle{Varieties}
    The notation \(V: YZ = X^2\) defines \(V\) as the set of points in \(\mathbb{P}^2\) satisfying \(YZ = X^2\). i.e. \(V = \{[X:Y:Z]\in \mathbb{P}^2 \mid YZ = X^2 \}\).
\begin{itemize}
    \item[\textrightarrow] \(V\) is what we call an algebraic set. In this case, it is also called a variety.
    \item[\textrightarrow] \(\mathbb{P}^n\), as well, is a variety for any \(n\in \mathbb{Z}^+\).

\end{itemize}
\end{frame}


\begin{frame}
    \frametitle{Functions}

    How do we understand functions on \(\mathbb{P}^2\)?
    \begin{itemize}
\item[\textrightarrow]
    We say a polynomial \(f\in \mathbb{C}[X, Y, Z]\) is homogenous of degree \(d\) if each term has the same degree \(d\). Equivalently, \(f(\lambda(X, Y, Z)) = \lambda^d f(X, Y, Z)\) for any \(\lambda\in\mathbb{C}^\ast\).

\item[\textrightarrow] For example, \(YZ - X^2\) is homogenous, but \(YZ - 1\) and \(Y^3 - XZ\) are not.

\end{itemize}
\end{frame}
\begin{frame}
    \frametitle{Functions 2}
    We define the function field of \(\mathbb{P}^2\), denoted \(\mathbb{C}(\mathbb{P}^2)\), as the subfield of functions \( \frac{f}{g}\) in \(\mathbb{C}(X, Y, Z)\) where \(f\) and \(g\) are homogenous polynomials of the same degree.
    \begin{itemize}
        \item[\textrightarrow] These functions are well-defined as functions of \(\mathbb{P}^2\)
        
    \end{itemize}



\end{frame}

\begin{frame}
    \frametitle{Functions 3}
    Like with \(\mathbb{P}^2\), we wish to define a function field over \(V\). To do this, we essentially set \(YZ - X^2\) to \(0\) in \(\mathbb{C}(\mathbb{P}^2)\).
    \begin{itemize}
        \item[\textrightarrow] To do this, first take the elements of \(\mathbb{C}(\mathbb{P}^2)\)
        \item[\textrightarrow] Remove the elements \( \frac{f}{g}\) where \(g\) is a multiple of \(YZ - X^2\).
        \item[\textrightarrow] Set that two functions \( \frac{f_1}{g_1}\) and \( \frac{f_2}{g_2}\) are equivalent if \( \frac{f_1}{g_1} - \frac{f_2}{g_2}\) is a multiple of \(YZ - X^2\). %Essentially what we're doing here though, is we're modding out the remaining set by an equivalence relation
        
    \end{itemize}
       \end{frame}

\begin{frame}
    \frametitle{Rational Maps}
    A rational map is a map between two varieties. We define rational maps between varieties \(V_1\) and \(V_2\) by:
    \[\varphi : V_1 \to V_2, \quad \varphi = [f_0: \ldots: f_n]\] where the functions \(f_0, \ldots, f_n \in \mathbb{C}(V_1)\), \(V_2\in\mathbb{P}^n\), and for any point \(P\in V_1\) where \(f_0, \ldots, f_n\) are all defined, \(\varphi(P)\in V_2\).

    Example: Using the \(V\) from before, define \[\varphi: V \to \mathbb{P}^1, \quad \varphi = \left[ \frac{Y}{Z} : 1\right].\]
    \begin{itemize}
        \item[\textrightarrow] The point \([2 : 4 : 1]\in V\) maps to \(\left[4 : 1\right]\)
        
    \end{itemize}
   
\end{frame}

\begin{frame}
    \frametitle{Injections of Function Fields}
    We can use the map \(\varphi\) to define a map between function fields:

   \[\varphi^\ast : \mathbb{C}(\mathbb{P}^1) \to \mathbb{C}(V), \quad \varphi^\ast f = f \circ \varphi.\]
   \begin{center}
   \begin{tikzcd}[ampersand replacement = \&, sep = large]
V \arrow[r, "\varphi"] \arrow[rd, "\varphi^\ast f = f \circ \varphi"', dashed] \& \mathbb{P}^1 \arrow[d, "f"] \\
                                                                             \& \mathbb{C}                 
\end{tikzcd}
\end{center}

    %So if I take a function \(f\in\mathbb{C}(\mathbb{P}^1)\), \(\varphi^\ast f:V\to \mathbb{C}\) takes a point in \(V\), maps it to \(\mathbb{P}^1\) by \(\varphi\), then inputs it into \(f\), creating a function in \(\mathbb{C}(\mathbb{P}^1)\).
\begin{itemize}
    \item[\textrightarrow] \(\varphi^\ast \frac{X}{Y} = \frac{Y}{Z}\)
    \item[\textrightarrow] \(\varphi^\ast \frac{XY^2 - X^3}{Y^3} = \frac{YZ^2 - Y^3}{Z^3}\)
\end{itemize}
\end{frame}
\begin{frame}
    \frametitle{Injections of Function Fields 2}
    \begin{itemize}
        \item Our map \(\varphi^\ast\) takes any function in \(\mathbb{C}(\mathbb{P}^1)\) and maps \(X\mapsto Y\), \(Y\mapsto Z\)
        \item This is an injective homomorphism of fields that fixes \(\mathbb{C}\)
        \item No other rational map can induce this injection
    \end{itemize}
        Now, we generalize this

    
   % \textbf{Warning:} Note the difference between the variables \(X, Y\) in \(\mathbb{C}(\mathbb{P}^1)\) and \(X, Y, Z\) in \(\mathbb{C}(V)\).
    \end{frame}

    \begin{frame}
        \frametitle{Curves}
        First, we define \textbf{curves}.

        Essentially, a variety is a curve if its function field can be described in one free variable over \(\mathbb{C}\).

       The prior examples \(V\) and \(\mathbb{P}^1\) are curves. We state without proof that \[ \mathbb{C}(\mathbb{P}^1) \cong \mathbb{C}(V) \cong \mathbb{C}(x)\] 

    \end{frame}

    \begin{frame}
        \frametitle{General Claim}
        \begin{enumerate}
            \item Let \(\varphi: V_1\to V_2\) be a nonconstant rational map where \(V_1\) and \(V_2\) are curves. Then \(\varphi\) induces an injection of function fields \(\varphi: \mathbb{C}(V_2) \to \mathbb{C}(V_1)\) that fixes \(\mathbb{C}\).

            \item For each injection of function fields \(\iota : \mathbb{C}(V_2)\to \mathbb{C}(V_1)\) that fixes \(\mathbb{C}\), there is a unique nonconstant rational map \(\varphi: V_1 \to V_2\) such that \(\varphi^\ast = \iota\).

            \item Thus, there is a one-to-one correspondence between nonconstant rational maps \(V_1 \to V_2\) and injections \(\mathbb{C}(V_2) \to \mathbb{C}(V_1)\) that fix \(\mathbb{C}\).
    \end{enumerate}
    \end{frame}
\end{document}
